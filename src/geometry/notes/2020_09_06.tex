\section*{Wprowadzenie}
\subsection*{Założenia}
\begin{itemize}
    \item tylko liczby całkowite
        \begin{itemize}[label={\(\implies\)}]
            \item bez dzielenia
            \item bez funkcji trygonometrycznych
            \item bez liczenia odległości (można porównywać ich kwadraty)
        \end{itemize}
\end{itemize}
\subsection*{Obiekty i~oznaczenia}
\begin{itemize}
    \item \(P = (x, y)\) punkt
    \item \(\vec{v} = [x, y]\) wektor (niezaczepiony)
    \item prosta
\end{itemize}
Nie jest złym pomysłem odróżenienie wektorów od punktów na poziomie typów danych.
\subsubsection*{Jakie działania?}
\begin{itemize}
    \item odjąć od siebie dwa punkty, uzyskując wektor z~jednego do drugiego: \(B - A = \vec{AB}\)
    \item dodać wektor do punktu uzyskując przesunięty punkt: \((x, y) + [a, b] = (x + a, y + b)\)
\end{itemize}
\section*{,,Klocki'' algorytmiczne}
\subsection*{Iloczyn skalarny}
\begin{mathfig*}
    \coordinate (origin) at (0, 0);
    \coordinate (A) at (2.5, 1);
    \coordinate (B) at (0, 2.5);
    \drawangle*[thick, ForestGreen, angle radius=0.7cm]{A--origin--B}[\(\alpha\)];
    \drawvec (origin) -- (A) node[right]{\(v_1 = [x_1, y_1]\)};
    \drawvec (origin) -- (B) node[above]{\(v_2 = [x_2, y_2]\)};
    \fillpoint{origin};
\end{mathfig*}
\begin{equation*}
    \begin{split}
        v_1 \cdot v_2 &\coloneqq x_1 \cdot x_2 + y_1 \cdot y_2\\
            &= \len{v_1} \cdot \len{v_2} \cdot \cos{\color{ForestGreen} \alpha}
    \end{split}
\end{equation*}
\subsubsection*{Przypadki wartości iloczynu skalarnego w~zależności od wzajemnego położenia wektorów}
\begin{itemize}
    \item równoległe: \(\max\) (\(\cos{\color{ForestGreen} \alpha} = +1\))
    \item kąt ostry: \(> 0\)
    \item prostopadłe: \(= 0\)
    \item kąt rozwarty: \(< 0\)
    \item przeciwrównoległe: \(\min\) (\(\cos{\color{ForestGreen} \alpha} = -1\))
\end{itemize}
\subsubsection*{Zastosowania}
\begin{itemize}
    \item test prostopadłości
    \item test zgodności zwrotów
\end{itemize}
\subsubsection*{Jak dostać wektor prostopadły do danego?}
\begin{mathfigure*}
    \coordinate (origin) at (0, 0);
    \coordinate (v) at (1.25, 0.5);
    \coordinate (left) at (-0.5, 1.25);
    \coordinate (right) at (0.5, -1.25);
    \drawrightangle[ForestGreen, angle radius=0.3cm]{v--origin--left};
    \drawvec[Orange] (origin) -- (v) node[right]{\(v = [x, y]\)};
    \drawvec (origin) -- (left) node[above left]{\([-y, x]\) ,,w lewo''};
    \drawvec (origin) -- (right) node[below right]{\([y, -x]\) ,,w prawo''};
    \fillpoint{origin};
\end{mathfigure*}
\subsection*{Iloczyn wektorowy}
\begin{mathfig*}
    \coordinate (origin) at (0, 0);
    \coordinate (A) at (2.5, 1);
    \coordinate (B) at (0, 2.5);
    \drawangle*[thick, ForestGreen, ->,angle radius=0.7cm]{A--origin--B}[\(\alpha\)];
    \drawvec (origin) -- (A) node[right]{\(v_1 = [x_1, y_1]\)};
    \drawvec (origin) -- (B) node[above]{\(v_2 = [x_2, y_2]\)};
    \fillpoint{origin};
\end{mathfig*}
\begin{equation*}
    \begin{split}
        v_1 \cross v_2 &\coloneqq x_1 \cdot y_2 - y_1 \cdot x_2\\
            &= \len{v_1} \cdot \len{v_2} \cdot \sin{\color{ForestGreen} \alpha}
    \end{split}
\end{equation*}
(skierowane pole równoległoboku rozpiętego przez te dwa wektory)
\subsubsection*{Przypadki}
\begin{itemize}
    \item skręt ,,w~lewo'': \(> 0\) (\(\ccw\))
    \item (przeciw)równoległe: \(= 0\)
    \item skręt ,,w~prawo'': \(< 0\)
\end{itemize}
\subsubsection*{Zastosowania}
\begin{itemize}
    \item test równoległości
    \item sortowanie kątowe
\end{itemize}
\subsection*{Sortowanie kątowe}
\begin{mathfigure*}
    \coordinate (origin) at (0, 0);
    \coordinate (a) at (2.5, 0);
    \coordinate (b) at (-0.5, 1.5);
    \coordinate (c) at (-1.5, -1.5);
    \coordinate (d) at (0, -3);
    \coordinate (e) at (1, -1);
    \drawvec (origin) -- (a);
    \drawvec (origin) -- (b);
    \drawvec (origin) -- (c);
    \drawvec (origin) -- (d);
    \drawvec (origin) -- (e);
    \fillpoint{origin};
\end{mathfigure*}
Chcemy uzyskać na tych wektorach porządek cykliczny. Wydaje się, że wystarczyłoby podać \(\ccw\) jako operator porównania do \texttt{std::sort}, ale niestety może to spowodować błędy, gdyż funkcja sortująca oczekuje porządku liniowego. Wybieramy więc wektor \emph{bazowy}, dzielimy wzdłuż niego zbiór wektorów na półpłaszczyzny, a~następnie każdą z~nich sortujemy oddzielnie, jako operatora porównania używając \(\ccw\).
\begin{mathfigure*}
    \coordinate (origin) at (0, 0);
    \coordinate (a) at (2.5, 0);
    \coordinate (b) at (-0.5, 1.5);
    \coordinate (c) at (-1.5, -1.5);
    \coordinate (d) at (0, -3);
    \coordinate (e) at (1, -1);
    \draw[Orange, dashed] (-2, 2) -- (origin);
    \draw[Orange, dashed] (origin) -- node[near end, sloped, above]{baza} (3, -3);
    \drawvec (origin) -- (a);
    \drawvec (origin) -- (b);
    \drawvec (origin) -- (c);
    \drawvec (origin) -- (d);
    \drawvec (origin) -- (e);
    \fillpoint{origin};
    \draw[Orange] (2, 1) node {poprawny porządek \(\ccw\)};
    \draw[Orange] (-2.5, -2.25) node {poprawny porządek \(\ccw\)};
\end{mathfigure*}
\subsection*{Prosta (skierowana)}
\begin{mathfigure*}
    \draw (-4.5, -1.5) -- node[very near end, below right]{\(Ax + By = 0\)} (4.5, 1.5);
    \draw (-4.5, 0.5) -- node[very near end, below right]{\(Ax + By + C = 0\)} (4.5, 3.5);
    \draw (6.5, 1.75) node{\(A^2 + B^2 \neq 0\)};
    \fillpoint*{0, 0}[\((0, 0)\)][below];
\end{mathfigure*}
\subsubsection*{Relacja punkt -- prosta}
\begin{itemize}
    \item \(Ax + By + C = 0 \implies\) punkt leży na prostej
    \item \(Ax + By + C > 0 \implies\) punkt leży po ,,dodatniej stronie'' prostej
    \item \(Ax + By + C < 0 \implies\) punkt leży po ,,ujemnej stronie'' prostej
\end{itemize}
Wartość uzyskana po podstawieniu współrzędnych punktu do równania prostej \emph{nie jest faktyczną odległością}, ale jest \emph{liniową funkcją odległości} punktu od prostej, to znaczy jeżeli \(\abs{Ax_1 + By_1 + C} < \abs{Ax_2 + By_2 + C}\), to punkt \((x_1, y_1)\) leży bliżej prostej niż punkt \((x_2, y_2)\).
\subsubsection*{Wektor normalny prostej}
Przyjmijmy, że mamy punkt \((x, y)\) należący do prostej, czyli \(Ax + By + C = 0\). Rozważamy jego przesunięcie o~wektor \([B, -A]\). Uzyskujemy punkt o~współrzędnych \((x + B, y - A)\). Po podstawieniu do równania prostej otrzymujemy
\begin{equation*}
    A(x + B) + B(y - A) + C = Ax + AB + By - AB + C = Ax + By + C = 0
\end{equation*}
Zatem nowy punkt też leży na prostej. Oznacza to, że wektor \([B, -A]\) ma kierunek zgodny z~kierunkiem prostej. Biorąc wektor do niego prostopadły \textcolor{ForestGreen}{w~lewo} uzyskujemy \textcolor{ForestGreen}{wektor normalny} tej prostej, który wyznacza jej dodatnią stronę.
\begin{mathfigure*}
    \coordinate (origin) at (0, 0);
    \coordinate (directional) at (-1.5, -0.5);
    \coordinate (normal) at (-2, 1);
    \drawrightangle[ForestGreen]{origin--directional--normal};
    \draw (-4.5, -1.5) -- node[very near end, below right]{\(Ax + By + C = 0\)} (4.5, 1.5)
        node[sloped, very near start, below]{\color{Orange} ujemna}
        node[sloped, very near end, above]{\color{Orange} dodatnia};
    \drawvec (directional) -- (origin);
    \drawvec (directional) -- node[left]{\color{ForestGreen} wektor normalny} (normal);
    \fillpoint*{directional}[\((x, y)\)][below];
    \fillpoint*{origin}[\([B, -A]\)][below right];
\end{mathfigure*}
