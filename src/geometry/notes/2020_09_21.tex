\subsection*{Średnica zbioru punktów}
Największa możliwa odległość między dwoma punktami tego zbioru. To \emph{nie jest} średnica koła opisanego na tym zbiorze! Ze względu na dokładność będziemy wyznaczać kwadrat średnicy.
\subsubsection*{Punkty tworzące średnicę leżą w~wierzchołkach otoczki}
Na pewno leżą na brzegu, bo da się przedłużyć odcinek między punktami nieleżącymi na otoczce aż do przecięcia z~otoczką. Przypuśćmy, że jakiś odcinek realizuje średnicę. Chcemy zamienić odległość punkt-punkt na odległość punkt-prosta. Prowadzimy prostą prostopadłą przez któryś z~końców odcinka. Przypomnijmy sobie, że punkt leży na otoczce, jeżeli istnieje taka prosta przechodząca przez ten punkt, że wszystkie punkty zbioru leżą po jednej stronie tej prostej. Zaprzeczmy tę definicję do dowodu nie wprost: niezależnie jaką prostą wybierzemy, to po obu stronach będą leżały punkty zbioru. Zatem jeśli jeden koniec średnicy leży po jednej stronie, to coś musi leżeć po drugiej. Średnica była prostopadła do wybranej prostej, więc jest odległością. Zatem punkt leżący po drugiej stronie jest dalej. Sprzeczność.
\QED
\subsubsection*{Jak wyznaczyć średnicę?}
Rozważamy tylko punkty z~otoczki. Jeżeli punkty są rozrzucone w~miarę równomiernie, to \(h \approx \sqrt{n}\), więc jeśli sprawdzimy wszystkie pary, to będziemy mieli czas liniowy. Niestety w~ogólnym przypadku to nie działa. Zróbmy inaczej: weźmy jakiś wierzchołek otoczki i~zastanówmy się, który jest od niego najdalej. Niestety ternary search nie wyjdzie, bo funkcja odległości nie jest bitoniczna. Nawet jeśli zaczniemy średnicę ,,z~drugiej strony'', to ternary search również może podać błędny wynik. Spróbujmy jeszcze inaczej: rozważmy prostą opartą na boku otoczki (czyli zawiera dwa punkty z~otoczki). Punkt najdalszy od tej prostej dorzucamy do trójki. Okazuje się, że wśród takich trójek jest średnica. Dowód: dorysujmy proste prostopadłe do średnicy przechodzące przez jej końce. Teraz obracamy tę instalację aż któraś z~prostych oprze się o bok otoczki. Teraz ten przeciwległy punkt jest najdalszym punktem od prostej która się oparła, bo po drugiej stronie nie ma żadnych punktów (na początku ich nie było, bo punkt należał do otoczki). Odległość punktu na otoczce od prostej opartej o~bok otoczki jest funkcją bitoniczną, ponieważ kąty naszych skrętów rosną w~miarę jak idziemy po otoczce, więc przez jakiś czas są mniejsze od \(180\degree\), a~potem większe od \(180\degree\), a~nasza zmiana odległości zależy od kąta, jak wektor naszego ruchu tworzy z prostą. Każdy odcinek daje nam co najwyżej cztery pary kandydatów na końce średnicy, więc mamy maksymalnie \(4\h\) możliwych średnic.
\subsubsection*{Ternary search? A~komu to potrzebne?}
Jeśli bierzemy następną prostą, to wszystkie (może poza pierwszym) kąty zmalały. Zatem najdalszy punkt od kolejnej prostej nie może być bliżej niż najdalszy do poprzedniej. Więc szukamy nawinie, ale od miejsca gdzie ostatnio skończyliśmy. Pętle \texttt{while} obrócą się co najwyżej \(2h\) razy, czyli mamy liniowy algorytm na średnicę + \(O(n\log{n})\) na wyznaczenie otoczki.
\subsubsection*{I03}
Czy zbiór punktów da się podzielić na dwa podzbiory o~średnicy mniejszej niż cały zbiór? Trzeba rozdzielić pary tworzące średnice. Zatem jest to test na dwudzielność grafu na punktach realizujących średnice (wierzchołki to punkty które należą do średnic, a~krawędzie są między tymi, które tworzą średnicę).
