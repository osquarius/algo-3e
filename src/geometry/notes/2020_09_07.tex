\section*{Coś na kształt \textit{Principia geometria}}
\subsection*{Punkt}
\begin{equation*}
    P = (x, y) \in \real^2
\end{equation*}
\subsection*{Odcinek}
\subsubsection*{Definicja~1.}
Odcinek \(PQ\) to zbiór składający się z~punktów \(P\), \(Q\) i~wszystkich puntów \(R\) takich, że punkty \(P\) i~\(Q\) leżą po przeciwnych stronach prostej przechodzącej przez punkt \(R\) i~prostopadłej do prostej \(PQ\).
\begin{mathfig*}
    \coordinate (P) at (0, 0);
    \coordinate (Q) at (1, 2);
    \coordinate (R) at (0.5, 1);
    \coordinate (Pprime) at (-0.5, 1.5);
    \coordinate (Qprime) at (1.5, 0.5);
    \drawrightangle[ForestGreen, angle radius=0.4cm]{Q--R--Pprime};
    \draw[shorten <= -0.5cm, shorten >= -0.5cm] (P) -- (Q) node[pos=1.25, right]{\(Ax + By + C_1 = 0\)};
    \draw[shorten <= -1.5cm, shorten >= -1.5cm] (Pprime) -- (Qprime) node[pos=1.25, above right]{\(Bx - Ay + C_2 = 0\)};
    \fillpoint*{P}[\(P = (x_1, y_1)\)][above left];
    \fillpoint*{Q}[\(Q = (x_2, y_2)\)][above left];
    \fillpoint*{R}[\(R\)][below];
\end{mathfig*}
\subsubsection*{Definicja~2.}
Odcinek \(PQ\) to zbiór punktów \(R\) takich, że
\begin{gather*}
    \vec{RP} \cdot \vec{RQ} < 0\\
    \vec{RP} \cross \vec{RQ} = 0
\end{gather*}
\begin{mathfig*}
    \coordinate (P) at (0, 0);
    \coordinate (Q) at (2, 1);
    \coordinate (R) at (1, 0.5);
    \drawvec (R) -- (P);
    \drawvec (R) -- (Q);
    \fillpoint*{P}[\(P\)][above left];
    \fillpoint*{Q}[\(Q\)][above right];
    \fillpoint*{R}[\(R\)][below right];
\end{mathfig*}
\subsubsection*{Definicja~3.}
Idziemy prostą skierowaną \(\vec{PQ}\) ruchem jednostajnym od nieskończoności. Wprowadźmy do definicji wymiar czasu \(t \in \real \cup \set{-\infty, +\infty}\) i~ustalmy, że w~punkcie~\(P\) chcemy znaleźć się w~chwili \(t = 0\), a~w~punkcie~\(Q\) w~chwili \(t = 1\). Wtedy naszą pozycję możemy wyrazić jako funkcję:
\begin{gather*}
    r\colon \real \mapsto \real^2\\
    \begin{split}
        r(t) &= P + \overbrace{(Q - P)}^{\text{wektor kierunkowy naszego ruchu}} \cdot \overbrace{t}^{\text{aktualny czas, skalar}}\\
            &= P + Q \cdot t - P \cdot t\\
            &= \underbrace{(1 - t) \cdot P + t \cdot Q}_{\text{punkty traktujemy tu wektorowo}}
    \end{split}
\end{gather*}
Wyrażenie w postaci
\begin{equation*}
    \alpha p + \beta q,\qquad \text{gdzie } \alpha + \beta = 1
\end{equation*}
nazywamy \highlight{kombinacją liniową} punktów \(p\)~i~\(q\). Gdy dodatkowo zachodzi warunek \(\alpha, \beta \in \closed{0}{1}\), mamy do czynienia z~ich \highlight{kombinacją wypukłą}. Zatem odcinek można zdefiniować jako zbiór wszystkich kombinacji wypukłych jego końców.
\subsection*{Ogólne definicje kombinacji}
Kombinacja liniowa zbioru punktów \(\set{p_1, p_2, \ldots, p_n} \subseteq \real^2\) to
\begin{equation*}
    \summation[i = 1][n]\alpha_ip_i,\qquad \text{gdzie } \summation[i = 1][n] \alpha_i = 1
\end{equation*}
Jeśli dodatkowo zachodzi warunek
\begin{equation*}
    \forall i \in \set{1, 2, \ldots, n}\colon \alpha_i \in \closed{0}{1}
\end{equation*}
mamy do czynienia z~kombinacją wypukłą.
\subsection*{Kiedy punkt leży wewnątrz trójkąta (o~niezerowym polu)?}
Aby to ustalić, uporządkujmy wierzchołki trójkąta \(\triangle ABC\) cyklicznie:
\begin{mathfigure*}
    \coordinate (A) at (0, 0);
    \coordinate (B) at (3, 2);
    \coordinate (C) at (-1, 2.5);
    \coordinate (Z) at (1, 1.5);
    \drawvec (A) -- (B);
    \drawvec (B) -- (C);
    \drawvec (C) -- (A);
    \fillpoint*{A}[\(A\)][below];
    \fillpoint*{B}[\(B\)];
    \fillpoint*{C}[\(C\)];
    \fillpoint*{Z}[\(Z?\)];
\end{mathfigure*}
\noindent
Punkt leży wewnątrz trójkąta wtedy i~tylko wtedy, gdy leży na jednym z odcinków \(AB, BC, CD\) lub leży po ,,takiej samej'' stronie (dla wszystkich po dodatniej albo dla wszystkich po ujemnej) prostych skierowanych \(\vec{AB}, \vec{BC}, \vec{CD}\). Formalnie:
\begin{equation*}
    \begin{split}
        Z \in \triangle ABC\\
        \verticaliff\\
        Z \in AB \quad &\lor \quad Z \in BC \quad \lor \quad Z \in CA\\
            &\lor \quad \sgn{\parens{\vec{AB} \cross \vec{AZ}}} = \sgn{\parens{\vec{BC} \cross \vec{BZ}}} = \sgn{\parens{\vec{CA} \cross \vec{CZ}}}
    \end{split}
\end{equation*}
\subsection*{Trójkąt to zbiór kombinacji wypukłych swoich wierzchołków}
Na pewno? Dowód:
\begin{itemize}
    \item[,,\(\implies\)''] każda kombinacja wypukła wierzchołków należy do trójkąta:
\end{itemize}
\subsection*{Wielokąt wypukły}
Wielokąt wypukły to suma zbiorów kombinacji wypukłych swoich wierzchołków.
\subsection*{Figura wypukła}
Dwie definicje tego, że figura \(F \subseteq \real^2\) jest wypukła:
\begin{enumerate}[label={(\arabic*)}]
    \item \(\forall P, Q \in F\colon \segment{PQ} \subseteq F\)
        \begin{mathfigure*}
            \draw (0, 0) node[below]{wypukła} -- (3, 1) -- (2, 3) -- (-1, 4) -- (-2, 3) -- (-1, 1) -- cycle;
            \draw[xshift=6cm] (0, 0) node[below]{nie jest wypukła} -- (3, 1) -- (2, 3) -- (-1, 4) -- (-2, 3) -- (-1, 1) -- (1, 2) -- cycle;
            \draw[ForestGreen] (1, 1) -- (1.5, 2);
            \draw[ForestGreen] (-1, 3) -- (0, 2.5);
            \draw[ForestGreen] (0.5, 0.3) -- (-1, 2);
            \draw[ForestGreen, xshift=6cm] (1, 1) -- (1.5, 2);
            \draw[ForestGreen, xshift=6cm] (-1, 3) -- (0, 2.5);
            \draw[Red, xshift=6cm] (0.5, 0.3) -- (-1, 2);
            \fillpoint[ForestGreen][ForestGreen]{1, 1};
            \fillpoint[ForestGreen][ForestGreen]{1.5, 2};
            \fillpoint[ForestGreen][ForestGreen]{-1, 3};
            \fillpoint[ForestGreen][ForestGreen]{0, 2.5};
            \fillpoint[ForestGreen][ForestGreen]{0.5, 0.3};
            \fillpoint[ForestGreen][ForestGreen]{-1, 2};
            \fillpoint[ForestGreen, xshift=6cm][ForestGreen]{1, 1};
            \fillpoint[ForestGreen, xshift=6cm][ForestGreen]{1.5, 2};
            \fillpoint[ForestGreen, xshift=6cm][ForestGreen]{-1, 3};
            \fillpoint[ForestGreen, xshift=6cm][ForestGreen]{0, 2.5};
            \fillpoint[Red, xshift=6cm][Red]{0.5, 0.3};
            \fillpoint[Red, xshift=6cm][Red]{-1, 2};
        \end{mathfigure*}
    \item \(F = \conv{F}\), gdzie \(\conv{F} \coloneqq\) zbiór punktów będących kombinacjami wypukłymi dowolnej skończonej liczby punktów z~\(F\)
\end{enumerate}
Pokażemy teraz, że te dwie definicje są sobie równoważne:
\begin{itemize}
    \item[(1) \(\implies\) (2)] Zauważmy, że tezę \(F = \conv{F}\) możemy rozbić na dwa warunki konieczne:
        \begin{enumerate}[label={\alph*)}]
            \item \(F \subseteq \conv{F}\), co jest oczywiste, ponieważ skończona liczba punktów to w~szczególności jeden punkt, a~każdy pojedynczy punkt jest swoją własną kombinacją wypukłą (ze współczynnikiem równym \(1\)). Zatem każdy punkt figury \(F\) należy do \(\conv{F}\).
            \item \(F \supseteq \conv{F}\), czyli że każdy punkt \(\conv{F}\) należy do figury \(F\). Możemy to pokazać indukcją po liczbie punktów branych do kombinacji wypukłej (niech będzie to liczba \(n\)):
                \begin{induction}
                    \item \(n = 1\): kombinacją wypukłą jednego punktu jest zawsze on sam, więc teza zachodzi
                    \item Załóżmy, że 
                \end{induction}
        \end{enumerate}
\end{itemize}
