\subsection*{Pokrycie wierzchołkowe (vertex cover)}
Przypomnienie: jeśli \(A\) jest zbiorem, to przez
\begin{equation*}
    \binom{A}{k}
\end{equation*}
rozumiemy zbiór wszystkich dokładnie \(k\)-elementowych podzbiorów \(A\).
\begin{description}
    \item[in:] \(G = \pars{V, E} \qquad E \subseteq \binom{V}{2}\)
    \item[out:] \(C \subseteq V\)
        \begin{equation*}
            \forall e \in E\colon e \cap C \neq \emptyset
        \end{equation*}
        Jest to ciekawy problem, gdy chcemy \(\min\card{C}\)
\end{description}
\subsubsection*{Twierdzenie Königa}
W~grafie dwudzielnym najmniej liczne pokrycie wierzchołkowe ma liczność równą liczności skojarzenia najliczniejszego. Zauważmy, że jeśli mamy dowolne pokrycie \(C\) i~skojarzenie \(M\), to
\begin{equation*}
    \card{C} \geq \card{M}
\end{equation*}
Ponieważ jeśli zdefiniujemy funkcję
\begin{equation*}
    f\colon M \mapsto C
\end{equation*}
to jest ona injekcją. Zatem moc dziedziny jest nie większa od mocy przeciwdziedziny. Czyli
\begin{equation*}
    C^*, M^*, \card{C^*} = \card{M^*} \implies C^* \text{ najmniej liczne i~} M^* \text{ najliczniejsze}
\end{equation*}
